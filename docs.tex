\documentclass{article}
\usepackage[margin=1in]{geometry}
\usepackage{listings}
\usepackage{hyperref}
\title{Chess Game Project Documentation}
\date{}

\begin{document}
\maketitle

\section{Overview}
This project implements a graphical chess game in C using SDL2. It
includes a minimax based AI, drag--and--drop movement, multiple color
themes and PGN save/load support. The helper script
\texttt{run\_game.sh} installs required packages, builds the project and
launches it.

\section{File Structure}
\begin{description}
  \item[include/] Header files for the board logic, engine and GUI.
  \item[src/] C sources implementing those components and the main entry
  point.
  \item[Makefile] Build rules using \texttt{gcc} and \texttt{sdl2-config}.
  \item[run\_game.sh] Script that installs SDL2 libraries with the
  available package manager, compiles the program and runs it.
  \item[docs.txt] The original plain text manual.
  \item[docs.tex] This LaTeX documentation.
\end{description}

\section{Building and Running}
\subsection{Using Make}
From the project directory execute:
\begin{lstlisting}[language=bash]
make clean && make
./chess_game
\end{lstlisting}

\subsection{Using \texttt{run\_game.sh}}
The script detects \texttt{apt}, \texttt{dnf} or \texttt{pacman} and
installs SDL2 libraries automatically:
\begin{lstlisting}[language=bash]
bash run_game.sh
\end{lstlisting}
If you encounter ``permission denied'' make it executable first:
\begin{lstlisting}[language=bash]
chmod +x run_game.sh
\end{lstlisting}

\section{Game Logic}
The chess rules are implemented in \texttt{board.c}. The\newline
\texttt{GameState} structure stores piece placement, castling rights,
turn, half--move clock and full move number. Move generation validates
legal moves and detects check, mate or draw. The move history allows
undo/redo and PGN saving.

\section{AI Engine}
\texttt{engine.c} provides a minimax search with alpha--beta pruning.
Evaluation uses piece--square tables and basic heuristics such as
material, mobility and king safety. Difficulty levels adjust the search
depth.

\section{Graphical Interface}
The SDL based interface lives in \texttt{gui.c}. Features include:
\begin{itemize}
  \item A \(1200\times800\) window with a \(640\,\mathrm{px}\) board.
  \item Multiple themes with improved contrast: classic, alt, neon and pastel.
  \item Drag--and--drop piece movement.
  \item Settings menu to choose AI difficulty and theme.
  \item Captured piece display with a material points panel beneath the move history.
  \item Pawn promotion dialog allowing choice of Queen, Rook, Bishop or Knight.
\end{itemize}

\section{Saving and Loading}
Games can be saved and loaded from PGN format. The loader strips headers,
comments and variations so a wide range of PGN files can be replayed.

\section{Extending}
Possible future improvements include stronger AI heuristics, networked
multiplayer and additional visual themes.

\end{document}
